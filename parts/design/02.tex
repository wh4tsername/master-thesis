\subsection{YT executor}

Далее мы рассмотрим специфику запуска именно на YT Map/Reduce кластере.

Api YT имеет возможность создания операций с многими входами, многими выходами. Причем зачастую, есть возможность писать выходные таблицы с промежуточных стадий составных операций.

Если рассмотреть некоторый подграф, состоящий целиком из ParDo, то теоретически ничто не мешает запустить его как одну Map операцию, которая внутри себя сохраняет структуру вызова пользовательских функций из ParDo. Наличие возможности указать многие входы и выходы операции позволяет запускать подграфы с несколькими истоками и стоками.

В свою очередь, GroupByKey в общем случае может быть представлен в виде MapReduce операции. Это объясняется тем, что перед функцией группировки будут вызвано какое-то количество ParDo преобразующих входные данные из YSON [ссылка] формата во внутреннее представление пары ключ-значение.

Существует важное замечание, что MapReduce - это операция Map, решардирование с помощью хэширования и запуск Reduce. Такого рода операция эффективнее Map, сортировки данных и Reduce.

I/O общение с таблицами кластера осуществляется через реализацию RawRead и RawWrite - YtRead и YtWrite. C помощью YtRead можно чтения таблиц в YSON или Protobuf [ссылка] форматах. YtWrite имеет возможность указания схемы, в том числе сортированной, для записи выходных значений. Так как рассматриваемая в данной работе реализация является proof-of-concept, мы опускаем реализации I/O в формате Protobuf и сортированных данных.
