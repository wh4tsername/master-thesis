\subsection{Тестирование корректности инструмента}

В первую очередь были написаны синтетические тесты, которые позволили убедиться, что model checker действительно перебирает все достижимые в исполнении теста состояния.

Первый тест (factorial) проверяет, что model checker перебирает все варианты доставки сообщений, одновременно находящихся в сети. Написан клиент, который на старте отправляет в сеть N сообщений с числами от 0 до N-1 и блокируется, ожидая все ответы. Серверная часть – RPC сервис, который имеет единственный метод, принимающий число и сохраняющий его в базу данных, а также логирующий значение. После запуска перебора мы должны получить все перестановки размера N.

Второй тест (ping-pong) – это пинг-понг между серверами: клиент отправляет запрос на один из серверов, этот сервер в свою очередь перенаправляет запрос другому серверу, тот отвечает таким же образом первому. Эта процедура повторяется пока счетчик выполненных запросов не дойдет до установленного лимита. Количество вершин в графе конфигураций вычисляется методом динамического программирования. Совпадение количества исследованных путей и ответа динамики означает, что посещены все состояния.

Следующий тест (livelock) иллюстрирует способность инструмента перебирать все состояния. В качестве сервера выступает сервис атомарной переменной с методами Load, Store, CompareExchange. Клиент реализует поверх такой атомарной переменной мьютекс, гарантирующий взаимное исключение,  но содержащий livelock. В случае графа конфигураций это означает существование бесконечного пути. Model checker находит этот путь и упирается на нем в ограничение глубины.

Последний тест (fetch-add) представляет собой клиента, реализующего операцию FetchAdd поверх сервиса атомарной переменной из прошлого теста через операцию CompareExchange. Одновременно в системе запускается несколько таких клиентов, каждый из которых делает один инкремент. От теста ожидается, что исполнение завершится и в атомарной переменной будет лежать начальное значение плюс количество пользователей. Можно запустить перебор всех состояний такого кода и сравнить суммарное число инкрементов с числом исполнений, умноженным на количество клиентов. В случае корректности model checker-а выполняется равенство.
